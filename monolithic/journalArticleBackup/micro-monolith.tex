%%%% Proceedings format for most of ACM conferences (with the exceptions listed below) and all ICPS volumes.
\documentclass[sigconf]{acmart}

\usepackage{booktabs} % For formal tables


% Copyright
%\setcopyright{none}
%\setcopyright{acmcopyright}
%\setcopyright{acmlicensed}
\setcopyright{rightsretained}
%\setcopyright{usgov}
%\setcopyright{usgovmixed}
%\setcopyright{cagov}
%\setcopyright{cagovmixed}


% DOI
\acmDOI{10.475/123_4}

% ISBN
\acmISBN{123-4567-24-567/08/06}

%Conference
\acmConference[COS700 Journal Ariticle]{COS700}{November 2017}{Pretoria, South Africa} 
\acmYear{2017}
\copyrightyear{2017}

\acmPrice{0.00}

\usepackage{url}
\usepackage{amssymb}
\usepackage{todonotes}
\begin{document}
\title{A Comparison of Implementation issues of Monolithic vs Microservices Software Architectures}
%\titlenote{Produces the permission block, and copyright information}
%\subtitle{Extended Abstract}
%\subtitlenote{The full version of the author's guide is available as \texttt{acmart.pdf} document}

\author{Nokuthula Manana}
\affiliation{%
  \institution{Department of Computer Science, University of Pretoria}
  \city{Pretoria} 
  \country{South Africa}
}
\email{u12064115@tuks.co.za}
\email{siphokazi.manana@gmail.com}

\author{Vreda Pieterse}
\affiliation{%
  \institution{Department of Computer Science, University of Pretoria}
  \city{Pretoria} 
  \country{South Africa}
}
\email{vpieterse@cs.up.ac.za}



\begin{abstract}
Microservices Architecture is a relatively new architecture upon which Software Engineers can build software. The whole premise of the architecture is to divide the different components of a system into small independently maintained modules in order to improve the testing and maintenance of a system. Microservices have been the buzz word in the industry for the past few years and some curious developers have been exploring this architecture as opposed to the more traditional monolithic architecture whereby the entire system is built and deployed at once, modules are inter-liked and upgrades to any part of the system would require the system to be reworked entirely. As the number of Software development companies in the world increase, it would be prudent to explore the advantages of microservices as an architecture and to what extent software engineering practices can benefit from its application. This article investigates the feasibility of using a microservices architecture as opposed to the more traditional monolithic architecture for software development.
\end{abstract}

%
% The code below should be generated by the tool at
% http://dl.acm.org/ccs.cfm
% Please copy and paste the code instead of the example below. 
%
\begin{CCSXML}
<ccs2012>
<concept>
<concept_id>10002944.10011123.10010912</concept_id>
<concept_desc>General and reference~Empirical studies</concept_desc>
<concept_significance>500</concept_significance>
</concept>
<concept>
<concept_id>10011007.10010940.10011003.10011002</concept_id>
<concept_desc>Software and its engineering~Software performance</concept_desc>
<concept_significance>500</concept_significance>
</concept>
</ccs2012>
\end{CCSXML}

\ccsdesc[500]{General and reference~Empirical studies}
\ccsdesc[500]{Software and its engineering~Software performance}
\keywords{microservices architecture, monolithic architecture}

\maketitle

\section{Introduction}
\input{introduction}

\section{Background}
\input{background}

%\section{Related work}
%\input{relatedwork}

\section{Research design}
\input{researchdesign}

\section{Data analysis}
\input{dataanalysis}

\section{Conclusion}
\input{conclusion}

\section{Future work}
\input{futurework}

%\input{body}

\bibliographystyle{ACM-Reference-Format}
\bibliography{micro-monolith} 

\end{document}
